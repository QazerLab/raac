\documentclass[letterpaper, 11pt]{article}

\usepackage[top=0in, bottom=1in, left=1in, right=1in, includehead, includefoot]{geometry}
\usepackage[T2A]{fontenc}
\usepackage[utf8]{inputenc}
\usepackage[english,russian]{babel}
\usepackage{titlesec}
\usepackage{xcolor}
\usepackage{graphicx}
\usepackage{wrapfig}
\usepackage{float}
\usepackage{hyperref}
\usepackage{amssymb}
\usepackage{enumitem}



\definecolor{headerColor}{HTML}{003300}

\titleformat{\section}[hang]
    {\Large}
    {\thesection}
    {0.3in}
    {}
    [\color{headerColor}\hrule height 0.5pt]

\setlist[itemize]{leftmargin=*}



\begin{document}

    \noindent
    \Huge
    Кивер Данила Андреевич \\ \\
    \normalsize
    danila.kiver@mail.ru или +7-909-645-38-84 \\

    \noindent
    Образование: МГУ имени М. В. Ломоносова, физический факультет, кафедра компьютерных методов физики, бакалавр (2015). \\

    \noindent
    Коммуникация: upper-intermediate английский; есть опыт работы в англоязычных командах; есть опыт работы в распределенных командах с разбросом во временных зонах до 10 часов. \\

    \noindent
    Публичные профили (кликабельно): \href{https://stackoverflow.com/users/7191047/danila-kiver}{StackOverflow}, \href{https://unix.stackexchange.com/users/297621/danila-kiver}{Unix StackExchange}, \href{https://github.com/QazerLab}{GitHub}.





    \section{Опыт работы}

    Общий стаж --- 9 лет.

    \begin{itemize}
        \item
            \textbf{Творческий отпуск} \\
            \footnotesize
                январь 2024 -- н. в., в стаж не входит
            \normalsize

            \begin{itemize}
                \item
                    Путешествия, языки, книги, pet-проекты и расширение технического кругозора.
            \end{itemize}

        \item
            \textbf{Старший разработчик, команда Архитектура} \\
            \footnotesize
                МойСклад (\url{https://moysklad.ru} и \url{https://kladana.com}), июль 2020 -- декабрь 2023
            \normalsize

            \begin{itemize}
                \item
                    Спроектировал систему управления окружениями и деплойментами для всей компании с учетом нюансов существующих в компании процессов.
                \item
                    Спроектировал новый вид девелоперских окружений для тестирования изменений с реальными данными по отдельному шарду (\texttt{microstage}, известный внутри компании как \texttt{minipig}) и реализовал с нуля инструментарий для управления этими окружениями: подготовки данных для окружения, оркестрации деплоймента и т. д.; разгрузил основное \texttt{stage}-окружение.
                \item
                    Внедрил разделение сборочной фермы на зоны под разные виды задач для оптимизации распределения нагрузки.
                \item
                    Внедрил использование статического анализа для прикладных сборок (spotbugs / maven dependency analyzer / hadolint / shellcheck), перевел все интеграционные тесты на использование testcontainers.
                \item
                    Участвовал в переработке системы построения отчетов (вынесении отчетов из основной транзакционной БД в отдельную аналитическую БД) --- проводил эксперименты с разными технологиями с замерами и анализом результатов; оптимизировал производительность прототипа новой системы.
                \item
                    Выполнял архитектурный надзор за изменениями, проектируемыми и разрабатываемыми продуктовыми командами: ревьюил архитектурные проекты и реализации новых прикладных сервисов.
                \item
                    Проводил технические интервью senior-разработчиков и техлидов.
                \item
                    Ни разу не уронил прод, имея полные доступы \raisebox{-0.5\height}{\includegraphics[width=10mm]{src/trollface_pipe_monocle.jpg}}
            \end{itemize}

        \item
            \textbf{Senior Software Engineer} \\
            \footnotesize
                EPAM Systems, май 2018 -- декабрь 2019
            \normalsize
            
            \begin{itemize}
                \item
                    Проект PERF: мигрировал решение (Java-монолит на SpringBoot + PostgreSQL) в облако (VM с Ubuntu Server $\rightarrow$ GKE).
                \item
                    Проект Axway Syncplicity (\url{https://www.syncplicity.com}): разрабатывал и поддерживал бэкенд системы (message queue, content search, cloud storage) и сервисы, устанавливаемые on-premises (on-prem storage). Технический стек --- Java-сервисы на SpringBoot + MariaDB на VM с CentOS 7 в AWS. Legacy-сервисы на Scala + Play Framework.
                \item
                    Участвовал во внутренней программе менторинга в роли ментора.
                \item
                    Выступал на технических митапах, в т. ч. на внешнем (тема --- детали реализации Linux-контейнеров и используемые ими интерфейсы ядра).
            \end{itemize}

        \item
            \textbf{Software Engineer / Senior Software Engineer} \\
            \footnotesize
                NetCracker Technology, ноябрь 2014 -- май 2018
            \normalsize

            \begin{itemize}
                \item
                    Разработал прототип системы Zero Touch Provisioning для проекта SD-WAN. Технический стек --- Java-сервисы на SpringBoot + PostgreSQL в OpenShift Origin.
                \item
                    Разработал и поддерживал HA-дистрибутив PostgreSQL на базе фреймворка Patroni.
                \item
                    Мигрировал существующее решение (Java-монолит на WildFly + Spring + PostgreSQL) в облачное окружение (OpenShift Origin).
                \item
                    Развивал и поддерживал HA-инфраструктуру на базе Pacemaker.
                \item
                    Развивал и поддерживал низкоуровневые Java-библиотеки (persistence, history, platform abstraction) корпоративной платформы.
            \end{itemize}
    \end{itemize}





    \section{Основные технологии}

    \begin{itemize}
        \item
            Java (основной язык), Spring / SpringBoot, PostgreSQL.
        \item
            Linux: 9 лет в проде с RHEL-like дистрибутивами; 15+ лет по жизни со всем остальным. \\
            \footnotesize
            на рабочей станции \textbf{исключительно} Linux с максимально кастомизированной под себя конфигурацией; Windows и Mac не использую в качестве основной системы ни при каких условиях.
            \normalsize

        \item
            Ansible, Docker, K8s и OKD (включая администрирование).
        \item
            OpenStack, AWS.
    \end{itemize}





    \section{Мы совместимы?}

    Небольшой список из утверждений для проверки на культурную совместимость:

    \renewcommand{\labelitemi}{\checkmark}

    \begin{itemize}
        \item
            Разработчик \textit{решает проблемы пользователей}, а не слепо и механически пишет код по ТЗ.
            \\ \\
            Используемые инструменты, технологии и формализованное ТЗ --- не догма и могут меняться в ходе работы, если это решит проблемы пользователей \textit{лучше}.

        \item
            использование слоев абстракций не снимает с разработчика обязанности знать или быть готовым разобраться, что происходит внутри используемых технологий.
            \\ \\
            \href{https://www.joelonsoftware.com/2002/11/11/the-law-of-leaky-abstractions}{Абстракции текут}, и когда это происходит, зачастую проще залезть в исходники, чем отлаживать черный ящик;

        \item
            CI/CD --- это подход к организации процесса разработки, а не конкретные инструменты и факт их использования сам по себе;

        \item
            DevOps --- это культура и методология разработки, применяемая командой, а не выделенная должность разработчика пайплайнов и манифестов.
            \\ \\
            \href{https://continuousdelivery.com/2012/10/theres-no-such-thing-as-a-devops-team/}{There's No Such Thing as a ``Devops Team''};

        \item
            по указанным выше причинам корректное применение практик CI/CD и DevOps невозможно без надлежащего образа мышления в команде разработки, без глубокого понимания этой командой используемых инструментов, инфраструктуры под разрабатываемыми приложениями и процессов разработки;

        \item
            код, как и история проекта, пишется единожды, а читается сотни раз:

            \begin{itemize}
                \item
                производительный, но нечитаемый и неподдерживаемый код --- мусорный код;

                \item
                история проекта без грамотного разбиения на минимальные и атомарные коммиты со внятными commit message'ами --- мусорная история;
            \end{itemize}

        \item
            удаленка --- это когда никого не волнует, в какой точке планеты, как долго и зачем ты находишься, как часто и куда перемещаешься. Важен только продукт, который ты создаешь.
    \end{itemize}

    \renewcommand{\labelitemi}{\textbullet}

\end{document}
